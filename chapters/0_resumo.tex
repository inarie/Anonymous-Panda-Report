\chapter*{Resumo}
\justify
Os efeitos psicossociais da pandemia COVID-19 nas pessoas e nas suas famílias, bem como o impacto global na segurança financeira, habitação e desemprego, tiveram um tremendo impacto negativo na saúde mental e no bem-estar de todos nós. Este é especialmente o caso de pessoas que já sofriam de transtornos mentais, uma vez que o isolamento social dificultou a obtenção de cuidados de saúde mental adequados.

Soluções digitais que permitem que as pessoas procurem tratamentos, como sessões de psicologia on-line e outras terapias através de meios tecnológicos, foram desenvolvidas tendo em conta indivíduos com doenças de saúde mental. No entanto, as pessoas têm preocupações sobre a privacidade dos seus dados, bem como a segurança das informações relacionadas com os seus problemas de saúde mental.

Este trabalho usa tecnologia de computação gráfica de ponta para aumentar o anonimato, mantendo a expressividade em intervenções de saúde mental mediadas por computador. Adicionalmente, através da personalização de um avatar realista usando o Unreal Engine 4 (UE4), Metahumans, e Live Link, argumentamos se é possível preservar a empatia entre paciente e terapeuta.

\keywords{humano virtual \and empatia \and anonimato \and tecnologias de apoio \and saúde mental}