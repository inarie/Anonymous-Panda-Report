\chapter*{Resumo}
\justify
Soluções digitais que permitem às pessoas procurar tratamento, tais como terapias psicológicas online e outras terapias com recurso à tecnologia, foram desenvolvidas para ajudar indivíduos com distúrbios de saúde mental. Tais abordagens podem suscitar preocupações sobre a privacidade na utilização dos dados das pessoas e a segurança da informação sobre a sua saúde mental.

Este trabalho utiliza tecnologia de ponta em computação gráfica para desenvolver um sistema inovador capaz de aumentar o anonimato, mantendo simultaneamente a expressividade nas intervenções de saúde mental mediadas por computador. Segundo os nossos resultados preliminares, conseguimos personalizar um avatar realista usando Live Link, Metahumans, e Unreal Engine 4 (UE4) com a mesma profundidade emocional que uma pessoa real. Além disso, os resultados mostraram que a incapacidade dos avatares virtuais de se expressarem através do movimento das mãos deu a impressão de que estavam a agir de uma forma pouco natural.

Ao incluir a função de rastreio das mãos utilizando o Leap Motion Controller, conseguimos melhorar a nossa compreensão do uso prospetivo de avatares humanos virtuais e ultrarrealistas na terapia de videoconferência, ou seja, os estudos realizados ajudaram-nos a compreender como as expressões faciais e corporais são vitais e como a sua ausência é problemática na comunicação com os outros.

\keywords{humano virtual \and empatia \and anonimato \and tecnologias assistivas \and saúde mental \and online self-disclosure}