\section{State of the Art}

\subsection{Expressiveness and Realism of Virtual Avatars}
Face-to-face interactions are hard to replicate online; however, today's technology allows us to create avatars that can display an array of human emotions. These virtual persons are increasingly being employed to facilitate interactions in various contexts. From health care to education, we are beginning to see an increase in the use of virtual agents as a means for people to learn complex tasks or improve their knowledge and experience \cite{MAR18, SHO19}. One interesting area of research is using virtual avatars in health care. For example, Zalake et al. \cite{ZAL18} have shown that computer-generated simulations are effective in increasing trustworthiness. Their experiment aimed to understand if the appearance of a virtual avatar affects users' trust and willingness to engage in future encounters with them. Twelve adults participated in the study and interacted with a virtual avatar (physically represented by a computer-generated head and body). This avatar was dressed professionally in one setting, while the other was dressed informally. They looked at the participants' self-reported trust levels and eye fixation duration on the virtual human's face during the experiment. There was no statistical difference in eye contact or trust between the two test settings, indicating that the users' trust levels were unaffected by the physical appearance of the virtual avatars.

A recent study by Milcent et al. \cite{MIL19} aims to determine if recognizing basic emotions on an expressive virtual human is as effective as recognizing basic emotions on an actual human using expressive wrinkles and pupillary size. Furthermore, expressive wrinkles have an impact on emotion recognition. Even though the impact of pupillary size is not as strong, they believe that both are necessary for creating an expressive virtual person. 
Still, what if virtual humans can be more engaging than real ones? Gratch et al. \cite{GRA07} showed that a simple virtual avatar with positive listening feedback could elicit more effective rapport-like effects than human-to-human contact. Rapport techniques are a concept of psychology that allows us to establish a connection of tuning and empathy with another person by demonstrating that his or her point of view or values is understood and respected. Their study's rapport, on the other hand, is based on mindless feedback (i.e., it does not understand the meaning of the speaker's narrative). While this type of input can be rather useful, and help us understand the connection with virtual humans, it is insufficient for most virtual human applications since it does not provide accurate feedback.

According to Kang et al. \cite{KAN16}, presenting virtual humans via mobile phone video chat services does no seem to affect how people feel and relate with virtual humans in the place of a real person. The researchers were interested in whether making the background more or less realistic would make any difference. Their results indicate that having a realistic background had no significant effect on the perception of any variable while having no background made the virtual human seem less natural and more appealing to women but not men.

Regarding the same scope, Kang et al. \cite{KAN16A} conducted another experiment to understand if people were motivated to stay engaged longer in a conversation with a virtual human on their smartphones if it had high visual fidelity. The study results showed that low visual fidelity had little effect on engagement rates, and although people preferred a 3D animated character over a regular video character with less synchronized backchannels, they still expected the video character to be real because they believed it was a person's video.

On a different setting and employing high-fidelity avatars created using photogrammetry 3D scan methods, Latoschik et al. \cite{LAT17} were interested in how realistic avatars affect users in a virtual environment. There were two types of avatars in their experiment: realistic and abstract. The realistic avatars were created using the previously mentioned methods, while abstract ones were based on a wooden dummy. Both avatar representations were alternately applied to participating users and the virtual counterpart in dyadic social encounters to evaluate the impact of avatar realism on self embodiment and social interaction quality. Then, participants' acceptance of virtual body ownership (VBO) was tested for both types of avatars and whether the appearance of other people's avatars impacts the users' self-perception. The results suggest that when realistic avatars are used as avatars for others, they are perceived as considerably more human-like, resulting in a higher level of acceptance of VBO. There was also evidence that users' self-perception is affected by the design and appearance of other people's avatars.

Finally, Zelenskaya and Harvey \cite{ZEL19} looked at public reactions during an experiment to understand how the empathic potential of virtual avatars could be explored and how well they could engage live audiences. The investigation was facilitated through a low-budget multimedia setup that applied the latest real-time motion capture technology trends to allow a human performer to control a 3D character named Paige and communicate with a live audience. Paige seemed to create a great first impression and gained more attention after a few short encounters. Many visitors were afraid to engage in a deeper dialogue, but most were interested in meeting the talent behind the scenes. Finally, observational data collected at the end of the study revealed that some people prefer to build a human relationship before interacting with a virtual character.

\subsection{Anonymity Issues in Virtual Avatars}
Virtual worlds and online interactions have their own set of social dynamics. Users are not as visible or accountable for their actions in these areas as in face-to-face contexts, and they can choose to be anybody they want. This phenomenon can be highly beneficial for persons dealing with particular emotional challenges. It can provide them with a chance to express themselves in ways they would not ordinarily be able to in their daily lives. However, in these virtual environments, those with ill intent might exploit the lack of accountability, often known as the "disinhibition effect" \cite{WAN20, KUR18}. In today's world, this is a common occurrence. Because of the internet, children and adults have spoken inappropriately and even bullied others, claiming to be safe behind their screens while forgetting that they're just as vulnerable to harm. But, emphasizing the positive side of anonymity, can virtual humans promote verbal self-disclosure from socially anxious interactants? For example, Kang et al. \cite{KAN10} demonstrate that virtual humans significantly improved the behavior of socially anxious interactants. According to the authors, when given contingent non-verbal feedback from virtual beings (rapport agents), those with high social anxiety felt more behavioral rapport by communicating more speech and information than those with low social pressure. This shows that if interactants have high levels of social anxiety, rapport feelings may hinder personal self-disclosure, which needs further investigation in future research.

On the other hand, can virtual humans encourage people to share information? According to Lucas et al. \cite{LUC14}, virtual humans can improve readiness to reveal in a clinical interview scenario. They show that virtual humans can have this effect by giving patients the feeling that their responses aren't being judged. They infer that virtual human interviewers' ability to extract more honest answers derives from the perception that no one monitors or evaluates them throughout the interview. The only difference between frames was the impression that another human was watching reactions during the interview. These findings suggest that automated virtual persons could help overcome a significant barrier to obtaining reliable patient data. Others intruding on people's privacy is the last thing they desire. However, the tendency of self-disclosure is altering our lifestyles. Even if we don't like it, we can't stop expressing ourselves online. Understanding how and why users reveal personal information in online social spaces and their perceptions of online privacy is a crucial research agenda in human-computer interaction. In social Virtual Reality (VR), Maloney et al. \cite{MAL20} focus on the kind of information users reveal and who they disclose it to, and their concerns about self-disclosure. According to their research, users can display their emotions, experiences, and personal info in social VR. Participants recognized that handing away biometric data to better the system meant revealing personal information in social VR was an inescapable trade-off. Some respondents saw providing information as having no privacy implications, while others were concerned about the scope and content of what they supplied. According to the researchers, these insights will help them better understand self-disclosure and privacy concerns in social VR and future alternatives for creating safer and more pleasant social VR platforms.

Finally, in another context, avatars may combine the benefits of face-to-face communication with the anonymity of online text-based communication in virtual reality for teletherapy. Baccon et al. \cite{BAC19} seek to discover if immersive virtual reality (VR) is a suitable medium for encouraging self-disclosure and, as a result, effective communication. Their study is the first to compare the frequency of objective and perceived self-disclosures between dyads who spoke in face-to-face, virtual reality, or online text-based situations. The authors also present an entirely new paradigm of VR research and kick off a conversation about how to use VR to foster relationships. Their research suggests that using avatars in virtual reality could be as practical as face-to-face communication in boosting self-disclosure. As a result, virtual reality could be a valuable tool for teletherapy since avatars allow us to combine the advantages of face-to-face contact with the anonymity of over-the-top communication.

\subsection{Combining And Tradeoffs between Expressiveness and Anonymity}
We have various expression tools in today's online world, such as blogs and forums. But a lot of them are cumbersome or lack expressiveness. Avatars are an exciting and promising new way to express oneself and connect with others, as they provide imaginative worlds for people to make their own, bring new kinds of self-expression to the web and introduce better ways in which people can talk about themselves online. Virtual Youtubers (VTubers), the most popular sort of user-generated video channel on the internet, are an outstanding example of this. “Vtubing” originated in China and has moved to Japan, Taiwan, and Korea, although VTuber videos can be found worldwide. The most considerable distinction between a VTuber and a standard YouTuber is that their videos use "moving image and sound" technology. It is like a more advanced version of image macros in that people express themselves through colorful animated avatars rather than still images. As previously stated, Lu et al. \cite{LU21} confirm that Virtual YouTubers, or human-voiced virtual 2D or 3D avatars, are gaining popularity as live streamers in East Asia. Although previous research has found that many viewers desire real-life interpersonal interactions with real-person streamers, the authors conducted an interview study to better understand how viewers interact with VTubers and how they perceive the identities of the voice actors who voice the avatars. According to the research, virtual avatars provide various performance opportunities, resulting in diverse spectator expectations and interpretations of VTuber behavior. With this in mind, the actors discovered that spectators are more willing to overlook impractical or dumb behavior in real-life humans. As a result, they maintained the disembodiment of VTuber avatars from their voice performers on purpose. In light of this research, we can deduce that expressiveness and anonymity can be combined for the greater good.

But can we empathize with a computer-generated character, and how do similar affective responses differ between encounters with real people and interactions with a computer-generated figure? According to Kang et al. \cite{KAN10A}, recent research has shown that connecting with virtual persons can increase social interactions or aid develop social abilities among those who have problems forming social bonds. This study looked into the impact of avatar realism and the interactants anticipated future interaction (AFI) on self-disclosure in emotionally engaged and synchronous interaction. According to preliminary research, the interactants' linguistic behaviors revealed more personal information about themselves in virtual persons than in actual humans. On the other hand, the AFI of interactions did not affect self-disclosure, which contradicts previous studies using text-based interfaces.

In a similar context, but with the use of mobile as a means of communication, Kang et al. \cite{KAN13} investigate the impact of various avatar realism levels on the three dimensions of Social Copresence: Psychological Copresence, Social Richness of the Medium, and Interactant Satisfaction with Communication, as well as the social communication quality of using anonymous avatars during small-screen mobile audio/visual communications. Higher levels of avatar Kinetic Conformity and Fidelity resulted in enhanced perceived Social Richness of Medium, according to trial results with 196 individuals using a simulated mobile device with varied avatar visual and behavioral realism. More robust degrees of Psychological Copresence and Interactant Communication Satisfaction were connected with higher avatar Anthropomorphism. On the other hand, increased avatar anonymity resulted in lower levels of Social Copresence. These impacts were lessened when avatars' visual and behavioral realism levels were higher.

Finally, in a completely different context, this study examines the movement of virtual signers, focusing on the concept of physical mobility as well as the issue of anonymity. Bigand et al. \cite{BIG19} suggested that an avatar should be animated as organically as feasible from motion models to establish perceptual acceptance. This acceptance can be ensured by employing motion capture devices to capture and transform natural movements into an avatar. Non-linguistic information may be sent by precise systems, such as the signer's identity. To address this, the ideal answer would be motion models or procedures that could hide the signer's identity while preserving linguistic information and allowing for as natural a motion as possible. However, it would be difficult to determine the movement parameters that convey essence to manage them, and we are far from having the necessary expertise.

\subsection{Other Approaches}
Although not directly related to the context of this thesis, the following approaches have a significant role in understanding tools based on the use of avatars. One of the approaches is AVATAREX, a telexistence system based on virtual avatars. AVATAREX unites people in the same physical space with their virtual counterparts simultaneously. Koskela et al. \cite{KOS18} used an indoor prototype version of AVATAREX and a simple collaborative game to investigate how users experienced co-presence in a telexistence system based on virtual avatars and to assess the performance of AVATAREX on high-end smart glasses. According to their findings, users using virtual reality gear reported a stronger sense of co-presence than users wearing augmented reality gear. Surprisingly, users using augmented reality smart glasses reported a lower sense of co-presence than those using a tablet.

The other approach is virtual reality therapy facilitated by the use of avatars. Using a Microsoft Kinect camera in virtual settings, Benrachou et al. \cite{BEN20} proposed an innovative framework capable of collecting and replicating human-like gestures in real-time. Their project focuses on avatars monitoring human posture failures to enhance people's posture throughout recovery stages (victims of stroke or mobility impairment). These scenes incorporate actual motion capture data, which aids in the resolution of postural issues in people with impairments. Patients can participate in functional tasks that have been designed by a physiotherapist and recreated as a serious game in a VR-based application. These challenging games must be supervised by rehabilitation aides, who use gesture control to set the pace of their exercise regimen. In this situation, physiotherapists or rehabilitation aides should be present to ensure that the study is replicated as accurately as possible under favorable settings for patient rehabilitation. Experiments show that the proposed model is adaptable, and during the evaluation phase, they created various situations to allow for a more in-depth analysis of patients' movements via avatars.

\subsection{Reflection}
Although the use of virtual humans in mental health systems has some limitations, these same virtual humans can provide new ways of interaction for people who want to benefit from anonymity. One of the goals of this research is to use the virtual human as an alternate self in real time, that is, to create a mimicked image of the user and transmit it to others as if it were the user themselves, thereby benefiting from a high level of anonymity. While existing research on Virtual Humans mainly tries to simulate human behavior to create empathy with the user, our work presents a novel direction to the use of virtual humans in the mental health domain. Through the real-time use of a hyper-realistic avatar in video conferencing, it is possible to maintain the level of expressiveness while preserving anonymity, whereas messaging and voice calls do not allow. Furthermore, by establishing a gap between the real and virtual selves during a video call, we are able to maintain a level of realism and expressionism that other types of avatars used in the studies could not, except for the Paige avatar \cite{ZEL19}, which despite the fact that this work is unrelated to our research topic, it aims at investigating people's reactions during the experience as well as empathy and the ability to engage with a live audience, which in turn helps contribute to the exploration of these virtual humans.