\section{State of the Art}

\subsection{Expressiveness and Realism of Virtual Avatars}
Face-to-face interactions are hard to replicate online; however, today's technology allows us to create avatars that can display an array of human emotions. These virtual characters are increasingly being employed to facilitate interactions in various contexts \cite{NOW18}, from health-care to education, we are beginning to see an increase in the use of virtual agents as a means for people to learn complex tasks or improve their knowledge and experience \cite{MAR18, SHO19}. One interesting area of research is using virtual avatars in health-care. For example, Zalake et al. \cite{ZAL18} have shown that computer-generated simulations are effective in increasing trustworthiness. Their experiment aimed to understand if the appearance of a virtual avatar affects users' trust and willingness to engage in future encounters with them. Twelve adults participated in the study and interacted with a virtual avatar (physically represented by a computer-generated head and body). This avatar was formally dressed in one setting, while the other was informally dressed. They looked at the participants' self-reported trust levels and eye fixation duration on the virtual human's face during the experiment. There was no statistical difference in eye contact or trust between the two test settings, indicating that the users' trust levels were unaffected by the physical attire of the virtual avatars.

Also related to physiognomy, a recent study by Milcent et al. \cite{MIL19} aims to determine if recognizing basic emotions on an expressive virtual human (VH) is as effective as recognizing basic emotions on an actual human using expressive wrinkles and pupillary size. Furthermore, expressive wrinkles have an impact on emotion recognition. Even though the impact of pupillary size is not as strong, they believe that both are necessary for creating an expressive virtual person. 
Still, what if virtual humans can be more engaging than real ones? Gratch et al. \cite{GRA07} showed that a simple virtual avatar with positive listening feedback could elicit more effective rapport-like effects than human-to-human contact. Rapport techniques are a concept of psychology that allows us to establish a connection of tuning and empathy with another person by demonstrating that his or her point of view or values is understood and respected. The rapport in this study is based on irrational feedback, which means that the avatar does not understand the meaning of the speaker's narrative but exhibits support for the user's point of view through nods. While this type of input can be rather useful and help us understand the connection with virtual humans, it is insufficient for most VH applications since it does not provide accurate feedback. 

In a distinct context, but one that is nevertheless connected to physiognomy, Latoschik et al. \cite{LAT17} were interested in how realistic avatars affect users in a virtual environment. There were two types of avatars in their experiment: realistic and abstract. The realistic avatars were created using the previously mentioned methods, while abstract ones were based on a wooden dummy. To assess the effect of avatar realism on self embodiment and social interaction quality, both avatar representations were alternatively applied to participating users and the virtual counterpart in dyadic social interactions. Then, participants' acceptance of virtual body ownership (VBO) was tested for both types of avatars and whether the appearance of other people's avatars impacts the users' self-perception. The findings imply that realistic avatars are viewed as significantly more human-like when employed as representations for other persons, leading to a higher level of VBO acceptance. There was also evidence that users' self-perception is affected by the design and appearance of other people's avatars.

When it comes to presenting virtual avatars in realistic scenarios with high visual fidelity, Kang et al. \cite{KAN16} believe that presenting virtual humans via mobile phone video chat services does not seem to affect how people feel and relate with the VH in contrast with connecting to a real person. So, they were interested in whether making the background more or less realistic would make any difference. Their results indicate that having a realistic background had no significant effect on the perception of any variable while having no background made the virtual human seem less natural and more appealing to women but not men.

Regarding the same scope, Kang et al. \cite{KAN16A} conducted another experiment to understand if people were motivated to stay engaged longer in a conversation with a VH on their smartphones if it had high visual fidelity. The study results showed that low visual fidelity had little effect on engagement rates, and although people preferred a 3D animated character over a regular video character with less synchronized backchannels, they still expected the video character to be real because they believed it was a person's video.

Finally, in order to understand how the empathetic potential of virtual avatars could be explored and how well they could engage live audiences, Zelenskaya and Harvey \cite{ZEL19} also studied the responses of the general public during an experiment. The experiment was made possible by a low-cost multimedia set-up that utilized the most recent developments in real-time motion capture technology to enable a human actor to operate the 3D avatar Paige and interact with a live audience. Paige seemed to create a great first impression and gained more attention after a few short encounters. Many visitors were afraid to engage in a deeper dialogue, but most were interested in meeting the talent behind the scenes. Finally, observational data collected at the end of the study showed that some people would rather develop a human relationship first than communicate with a virtual character (VC).

\subsection{Anonymity Issues in Virtual Avatars}
Virtual worlds and online interactions have their own set of social dynamics. Users are not as visible or accountable for their actions in these virtual environments as in face-to-face contexts, and they can choose to be anybody they want. This phenomenon can be highly beneficial for persons dealing with particular emotional challenges, since it can provide them with a chance to express themselves in ways they would not ordinarily be able to in their daily lives \cite{LUC14, ROT19, BAT14}.

However, in these virtual environments, those with ill intent might exploit the lack of accountability, often known as the "disinhibition effect" \cite{WAN20, KUR18}. In today's world, this is a common occurrence. Because of the internet, children and adults have spoken inappropriately and even bullied others, claiming to be safe behind their screens while forgetting that they're just as vulnerable to harm. But, emphasizing the positive side of anonymity, can virtual humans promote verbal self-disclosure from people suffering from mental illnesses and related
disorders? For example, Kang et al. \cite{KAN10} demonstrate that VH significantly improved the behavior of socially anxious interactants. The authors found that people with high social anxiety had greater behavioral rapport and communicated more than those with low social pressure in response to contingent non-verbal input from virtual beings (rapport agents). This shows that if interactants have high levels of social anxiety, rapport feelings may hinder personal self-disclosure, which needs further investigation in future research.

On the other hand, can the anonymity associated with virtual humans encourage people to share information? According to Lucas et al. \cite{LUC14}, VH can improve readiness to reveal in a clinical interview scenario. They show that virtual humans can have this effect by giving patients the feeling that their responses are not being judged. They infer that VH interviewers' ability to extract more honest answers derives from the perception that no one monitors or evaluates them throughout the interview. The only difference between frames was the impression that another human was watching reactions during the interview. According to these results, a significant barrier to acquiring trustworthy patient data might be overcome with the aid of automated virtual persons. Others intruding on people's privacy is the last thing they desire. However, the tendency towards online self-disclosure is altering our lifestyles \cite{ELE14}. Even if we do not like it, we can not stop expressing ourselves online. Understanding how and why users divulge personal information in online social spaces, as well as their perceptions of online privacy, is an essential research agenda in human-computer interaction. 

In Social Virtual Reality (i.e. a type of online social interaction that uses immersive technology and takes place in three-dimensional virtual worlds where users, represented by avatars, may participate in real-time interpersonal dialogue and shared activities), Maloney et al. \cite{MAL20} focus on the kind of information users reveal and who they disclose it to, and their concerns about self-disclosure. According to their research, users can display their emotions, experiences, and personal info in Social Virtual Reality. Participants understood that providing personal information in Social Virtual Reality was an unavoidable trade-off for turning over biometric data to improve the system. Some respondents saw providing information as having no privacy implications, while others were concerned about the scope and content of what they supplied. According to the researchers, these findings will help them in better understanding self-disclosure and privacy concerns in Social Virtual Reality, as well as potential solutions for building safer and more pleasurable Social Virtual Reality platforms.

Finally, avatars in Virtual Reality (VR) for teletherapy may combine the benefits of face-to-face conversation with the anonymity of online text-based communication. Baccon et al. \cite{BAC19} attempted to determine if immersive virtual reality can encourage self-disclosure and, as a result, better communication. Their research is the first to contrast the frequency of objective and perceived self-disclosures amongst dyads that chatted in person, virtual reality, or online text-based circumstances. The authors also present an entirely new paradigm of VR research and kick off a conversation about how to use VR to foster relationships. Their research suggests that using avatars in virtual reality could be as practical as face-to-face communication in boosting self-disclosure. As a result, since avatars allow us to combine the benefits of face-to-face contact with the anonymity of over-the-top communication, VR could be a good tool for teletherapy.

\subsection{Combining And Tradeoffs between Expressiveness and Anonymity}
We have various expression tools in today's online world, such as blogs and forums. However a lot of them are cumbersome or lack expressiveness. Avatars offer imaginative worlds where people can be whatever they want to be, bring new forms of self-expression to the web, and create better ways to talk about themselves online. They are an exciting and promising new way to express yourself and connect with others. Virtual Youtubers (VTubers) are an outstanding example of this since they enable a customized animated virtual avatar that captures the user's facial movements and controls the avatar's facial expressions. “Vtubing” originated in China and has moved to Japan, Taiwan, and Korea, although VTuber videos can be found worldwide. The most considerable distinction between a VTuber and a standard YouTuber is that their videos use "moving image and sound" technology. It resembles a more advanced version of image macros in that people express themselves through colorful animated avatars rather than still images. As previously stated, Lu et al. \cite{LU21} confirm that Virtual YouTubers, or human-voiced virtual 2D or 3D avatars, are gaining popularity as live streamers in East Asia. Despite prior research indicating that many viewers want real-life interpersonal contacts with real-person streamers \cite{LU21, TUR22}, the authors performed an interview study to better understand how viewers interact with VTubers and how they interpret the identities of the voice actors that voice the avatars. According to the study, virtual avatars present a variety of performance opportunities, which leads to a wide range of spectator expectations and interpretations of VTuber behavior. With this in mind, the actors realized that audiences are more willing to tolerate impractical or foolish human behavior. As a result, they purposefully differentiated their VTuber avatars from their voice artists. Based on the findings of this study, we believe that, in settings other than streaming, expressiveness and anonymity can coexist. Thus, even though more unrealistic avatars promote tolerance for inappropriate behavior, they might also encourage ease on the mind of the avatar user.

But can we empathize with a computer-generated character, and, if so, how do similar affective responses differ between encounters with real people and interactions with a computer-generated figure? According to the latest study of Kang et al. \cite{KAN10A}, engaging with virtual characters can boost social interactions or help those who have difficulty developing social bonds develop social skills. The purpose of this study was to investigate the effect of avatar realism and interactants' anticipated future interaction (AFI)---condition that occurs when individuals who have had an initial interaction expect to meet again in the future---on self-disclosure in emotionally engaged and synchronous connections. Preliminary findings suggest that the linguistic behaviors of interactants revealed more personal information about themselves in virtual persons than in real humans. However, the AFI of interactions had no effect on self-disclosure, which contradicts prior research employing text-based interfaces.

In a similar context, but with the use of mobile as a means of communication, Kang et al. \cite{KAN13} investigate the effect of various avatar realism levels on the three types of social copresence—psychological copresence, social richness of the medium, and interactant satisfaction with communication—as well as the social communication quality of employing anonymous avatars during small-screen mobile audio/visual interactions. Their results show that, when using a simulated mobile device, hight levels of avatar kinetic conformity (the degree to which the avatar's dynamic movement reflects the expected motion of the object that the avatar appears to be) and fidelity (visual quality of the avatar image) resulted in an increase of perceived social richness of medium (the user's perception of the communication medium's social and emotional skills). More robust degrees of Psychological Copresence and Interactant Communication Satisfaction were connected with higher avatar Anthropomorphism. On the other hand, increased avatar anonymity resulted in lower levels of Social Copresence. These impacts were lessened when avatars' visual and behavioral realism levels were higher.

Finally, Bigand et al. \cite{BIG19} investigate the movement of virtual signers, focusing on the concepts of physical mobility and anonymity. The authors propose that an avatar be organically animated and feasible by utilizing motion models to establish perceptual acceptance. This acceptance can be ensured by employing motion capture devices to capture and transform natural movements into an avatar. Non-linguistic information may be sent by precise systems, such as the signer's identity. To address this, the ideal answer would be motion models or procedures that could hide the signer's identity while preserving linguistic information and allowing for as natural a motion as possible. However, it would be difficult to determine the movement parameters that convey essence to manage them, and we are far from having the necessary expertise.

\subsection{Other Approaches}
Although not directly related to the context of this thesis, the following approaches have a significant role in understanding tools based on the use of avatars. A telexistence system (system that allows a human user to operate in an environment without physically being there) based on virtual avatars called AVATAREX is one of the methods. AVATAREX connects persons who are in the same physical place with their virtual counterparts. Koskela et al. \cite{KOS18} investigated how users experienced co-presence in a telexistence system based on virtual avatars using an indoor prototype version of AVATAREX and a basic collaborative game, as well as the performance of AVATAREX on high-end smart glasses. Users wearing VR gear reported a better sense of coherence than users wearing augmented reality gear, according to their findings. Surprisingly, users of augmented reality smart glasses reported a lesser sense of co-presence than those who used a tablet.

The other approach is virtual reality therapy facilitated by the use of avatars. Using a Microsoft Kinect camera in virtual settings, Benrachou et al. \cite{BEN20} proposed an innovative framework capable of collecting and replicating human-like gestures in real-time. Their work is centered on avatars monitoring human posture failures in order to improve people's posture throughout the healing stages (victims of stroke or mobility impairment). These scenarios include real-world motion capture data, which aids in the resolution of postural difficulties in patients at different stages of rehabilitation. Patients could take part in functional tasks that a physiotherapist designed and replicated as a serious game in a VR-based application. These demanding games were overseen by rehabilitation aids, who used gesture control to set the speed of their exercise regimen. In this situation, physiotherapists or rehabilitation aides should be present to ensure that the study is replicated as accurately as possible under favorable settings for patient rehabilitation. Experiments show that the proposed model is adaptable, and they were able to created various situations in order to allow an in-depth analysis of patients' movements via avatars.

Finally, regarding virtual human support agents, Rizzo et al. \cite{RIZ11} created the SimCoach project, which intends to construct support agents to serve as online guides for encouraging access to psychological health-care information and supporting military members and relatives in breaking down barriers to seek health-care. The experience was designed to drive users to take the initial step and to urge them to take the next step toward finding other, more official resources if necessary. However, while the VH were supposed to use speech, gesture, and emotion to explain their capabilities, solicit basic anonymous background information, provide advise and support, lead the user to relevant online content, and potentially simplify the process of obtaining appropriate care with a real clinical professional, their engagement with the user is limited because they receive input through chat messages.

\subsection{Reflection}
Although the use of virtual humans in mental health systems has still some limitations, such as the use of VH with mindless feedback \cite{GRA07}, VH with written feedback \cite{RIZ11}, and the use of low fidelity avatars \cite{LU21,KAN10, KAN10A, BAC19}, they still can provide new ways of interaction for people who want to benefit from anonymity. One of the goals of this dissertation is to create a mimicked image of a user and convey it to others as if it were the user themselves by employing a virtual human as an alternate self in real-time. While existing research on virtual humans mainly tries to simulate human behavior to create empathy with the user, our work takes this one step further by using cutting-edge technology to ensure anonymity and promote self-disclosure in the mental health domain. Through the real-time use of a ultra-realistic avatar in video conferencing, we argue if it is possible to maintain the level of expressiveness while preserving anonymity. Furthermore, by establishing a gap between the real and virtual selves during a video call, we are able to maintain a level of anonymity and expressionism that other forms of avatars used in other studies could not, with the exception of the Paige avatar \cite{ZEL19}, which aimed to investigate people's reactions during their experience as well as empathy and the ability to engage with a live audience, which in turn contributed in the investigation of virtual humans.