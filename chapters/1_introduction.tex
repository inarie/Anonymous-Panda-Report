\section{Introduction}

\subsection{Motivation}
In the last few years, people have been increasingly struggling with mental illnesses and related disorders \cite{COS20, KUM21}. Many of todays' cases might be related to the COVID-19 global pandemic. While the economic impact of social isolation and quarantine has been estimated, the number of infections has continued to climb, with cases and deaths rising in many places \cite{CLA20, GAY20, RAS21}. The disease's direct consequences on individuals and their families and its indirect implications on financial security, housing, unemployment, and social isolation have led to an increase in emotional and psychological issues around the world, with particular communities being disproportionately impacted \cite{CLA20, GAY20, RAS21}. Worry, stress, anxiety, and other emotional responses are common in times of uncertainty \cite{EHR20, CHE20, JOH20}, however they can intensify the symptoms of those already who have a mental illness. Another issue for people with mental illness during the pandemic is that social isolation makes getting appropriate mental health care more challenging. Furthermore, mental illness is associated with a cultural stigma, making people reluctant to seek care, which can increase feelings of loneliness and isolation \cite{HAM18}.

A convenient and growing solution, especially during the COVID-19 pandemic \cite{BOL20, BEK20, WIN20}, for supporting mental health patients is the use of videoconferencing systems. However, this convenience might be problematic for mental health because patients' privacy and anonymity must be maintained. Moreover, the patient himself could become more stigmatized with the videoconference technology \cite{BAT14, MAL22} and could feel less prone to expressing the feelings, leading to more significant difficulties from the therapeutic perspective. On the other hand, to maintain privacy and anonymity, the patients can resort to text therapy or even phone consultations \cite{AGY21, LIB21, AGU21, AGY20, ZHA20}. Patients and counselors may find these types of treatment convenient, but they are no match for face-to-face interactions since both patients and therapists do not have access to nonverbal cues, which are an important part of diagnosing the problem \cite{SUC12, LIN21}.

This dissertation presents the development of a working prototype capable of anonymously connecting people with mental health specialists to overcome the lack of confidentiality and security on videoconferencing therapy and the lack of expressiveness of text/phone therapy. With the introduction of ultra-realistic virtual human avatars, such as Unreal Engine's Metahumans \cite{EPI21, FAN21}, it is now possible to attain higher levels of anonymity while retaining the expressive powers of a standard videoconference.

\subsection{Contributions}
This work brings a two-folded contribution to both the virtual reality research field as well as mental health support technologies:

(i) a new videoconferencing prototype that tracks facial expressions and positions in 3D and realtime, and uses ultra-realistic human avatars, applied to the context of mental health video conferencing;

(ii) a new body of knowledge regarding peoples’ behavioral attitudes and intentions towards using such a system in the future, with particular interest in how people manage to balance anonymity and expressiveness.

\subsection{Research Questions}
In the pursuit of human happiness and well-being, it is critical to assist individuals in being more honest. This purpose has contributed to the development of a new type of assistive technology, one that employs virtual avatars (VA) to assist patients in opening up and connecting more deeply and honestly with mental health professionals. The following questions have arisen in order to comprehend the benefits of using VA in mental health therapy:

\textbf{[RQ\textsubscript{1}]} - Can virtual avatars elicit the same emotional response as human-to-human interactions?

\textbf{[RQ\textsubscript{2}]} - Can virtual avatars help promote interactants' verbal self-disclosure?

Answering these questions is important because they can help us understand how metahumans can support people in opening up and interacting with others more honestly.

Although other researchers have addressed questions similar to ours \cite{LU21, ZAL18, GRA07, LUC14, ROT19, KAN16, KAN10A, BAC19}, these research questions are essential since the VA used in this experiment are new and underexplored. Furthermore, they exhibit more realistic human traits than other works \cite{LU21, ZAL18, GRA07, LUC14, ROT19, KAN16, KAN10A, BAC19}, implying that the findings of similar works cannot be considered to be valid because the conditions are distinct.

\subsection{Document structure}
Following the contextual analysis, a literature review was developed, which contains subsections for expressiveness and realism of VA, anonymity issues in virtual avatars, the combining and tradeoffs between expressiveness and anonymity, and finally other approaches related to VA. When the state-of-the-art workflow was drafted, each subsection was compared and correlated with the proposed solution.

The first prototype is shown in Chapter III. This chapter describes how it was constructed and which metahumans were chosen, as well as how it works with video conferencing tools, how it identifies human facial expressions, and some of the prototype's limitations. Finally, it goes over the verification of the prototype's viability in great detail, including a description of the participants and procedure, as well as how the study was evaluated, providing results as well as some analysis of the findings.

Chapter IV shows a new updated prototype that emphasizes the role of gestures when speaking. This chapter goes into considerable detail about the implementation of hand tracking, as well as some of the prototype's issues and limits. Additionally, the new UI prototype was included. Last but not least, it covers in great detail how the viability of the prototype was verified. This includes a description of the participants and procedure, as well as how the study was evaluated, providing results and some analysis of those results. Then, Chapter V discusses the study's conclusions and future work.