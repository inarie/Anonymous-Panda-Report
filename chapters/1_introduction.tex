\section{Introduction}

\subsection{Motivation}
In the last few years, people have been increasingly struggling with mental illnesses and related disorders \cite{COS20, KUM21}. Many of todays' cases might be related to the COVID-19 global pandemic. While the economic impact of social isolation and quarantine has been estimated, the number of infections has continued to climb, with cases and deaths rising in many places. The disease's direct consequences on individuals and their families and its indirect implications on financial security, housing, unemployment, and social isolation have led to an increase in emotional and psychological issues around the world, with particular communities being disproportionately impacted \cite{CLA20, GAY20, RAS21}. Worry, stress, anxiety, and other emotional responses are common in times of uncertainty \cite{EHR20, CHE20, JOH20}, however they intensify the symptoms of those already who have a mental illness. Another issue for people with mental illness during the pandemic is that social isolation makes getting appropriate mental health care more challenging. Furthermore, mental illness is associated with a cultural stigma, making people reluctant to seek care, which can increase feelings of loneliness and isolation.

A convenient and growing solution, especially during the COVID-19 pandemic \cite{BOL20, BEK20, WIN20}, for supporting mental health patients is the use of videoconferencing systems. However, this convenience is sometimes problematic for mental health since patient privacy and anonymity need to be maintained. Moreover, the patient himself could become more stigmatized with the videoconference technology and could feel less prone to expressing the feelings, leading to more significant difficulties from the therapeutic perspective. On the other hand, to maintain privacy and anonymity, the patients can resort to text therapy or even phone consultation \cite{AGY21, LIB21, AGU21, AGY20, ZHA20}. Patients and counselors may find these types of treatment convenient, but they are no match for face-to-face interactions. The patient may not receive the therapy they require if the therapist disregards nonverbal cues that aid in diagnosing the problem \cite{SUC12, LIN21}.

The purpose of this project is the development of a working prototype capable of anonymously connecting people with mental health specialists to overcome the lack of confidentiality and security on videoconferencing therapy and the lack of expressiveness of text/phone therapy. With the introduction of ultra-realistic human avatars, such as Unreal Engine's Metahumans \cite{EPI21, FAN21}, it is now possible to attain higher levels of anonymity while retaining the expressive powers of a standard videoconference. This thesis will put this hypothesis to the test in various scenarios and will have an impact on a critical area in today's world.

\subsection{Contributions}
This work brings a two-folded contribution to both the virtual reality research field as well as mental health support technologies:

(i) a new videoconferencing prototype that tracks facial expressions and positions in 3D and realtime, and uses ultra-realistic human avatars, applied to the context of mental health video conferencing;

(ii) a new body of knowledge regarding peoples’ behavioral attitudes and intentions towards using such a system in the future, with particular interest in how people manage to balance anonymity and expressiveness.

\subsection{Research Questions}
In the pursuit of human happiness and well-being, it is critical to assist individuals in being more honest. This purpose and the global COVID-19 pandemic have aided in the development of a new type of supportive technology, one that employs virtual avatars to assist patients in opening up and connecting more deeply and honestly with mental health professionals. The following questions have arisen in order to comprehend the benefits of using virtual avatars in mental health therapy:

\textbf{[RQ1]} - Can virtual avatars elicit the same emotional response as human-to-human interactions?

\textbf{[RQ2]} - How can virtual avatars help promote verbal self-disclosure from socially anxious interactants?


% TODO - Change document structure

\subsection{Document structure}
Following the contextual analysis, a literature review was developed, which contains subsections for expressiveness and realism of virtual avatars, anonymity issues in virtual avatars, the combining and tradeoffs between expressiveness and anonymity, and finally other approaches related to virtual avatars. When the state-of-the-art workflow was drafted, each subsection was compared and correlated with the proposed solution.

The initial prototype is presented in Chapter III. This chapter includes information on how it was created and which metahumans were chosen, as well as details on how it works with video conferencing tools and how it recognizes human facial expressions. Following that is a section that discusses some of the current prototype's limitations as well as some future challenges.

Chapter IV goes over the verification of the initial prototype's viability in great detail. Finally, in the following section, the participants and procedure, as well as how the evaluation of the study will be done.