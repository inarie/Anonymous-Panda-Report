\chapter*{Abstract}
\justify
Digital solutions that allow people to seek treatment, such as online psychological interventions and other technology-mediated therapies, have been developed to assist individuals with mental health disorders. Such approaches may raise privacy concerns about the use of people's data and the safety of their mental health information.

This work uses cutting-edge computer graphics technology to develop a novel system capable of increasing anonymity while maintaining expressiveness in computer-mediated mental health interventions. According to our preliminary findings, we were able to customize a realistic avatar using Live Link, Metahumans, and Unreal Engine 4 (UE4) with the same emotional depth as a real person. Furthermore, these findings showed that the virtual avatars' inability to express themselves through hand motion gave the impression that they were acting in an unnatural way.

By including the hand tracking feature using the Leap Motion Controller, we were able to improve our comprehension of the prospective use of ultra-realistic virtual human avatars in videoconferencing therapy, i.e., both studies helped us understand how vital facial and body expressions are and how problematic their absence is in communicating with others.

\keywords{virtual human \and empathy \and anonymity \and assistive technologies \and mental health \and online self-disclosure}